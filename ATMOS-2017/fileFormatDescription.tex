\documentclass[11pt,paper=a4]{article}

\usepackage{csquotes}


\begin{document}


\section*{Data Format Description: Mobility Offer Instances}

In the following, a description of an input data format for
mobility offer problems is given.
All indices are zero based, i.e the index~0 refers to the first element of a list.
The value~$-1$ for an index encodes that an optional index is not present.
The input data is provided in a plain text file
with the file ending \enquote{{\bf .mo.input}} that consists of:

\begin{itemize}
	\item One line containing two natural numbers
		  separated by single spaces:
		\begin{itemize}
			\item \emph{nmbVehicles} The number of vehicles
			\item \emph{nmbDemands} The number of demands
		\end{itemize}

	\item \emph{nmbDemands} subsequent lines each describing a mobility demand.
		The first number of each line specifies the \emph{order date} of the demand.
		The second number of each line specifies the \emph{number of mobility offers}~$n$ for the mobility demand.
		It is followed by $n$~five-tuples of numbers, specifying
		\emph{cost},
		\emph{interval start},
		\emph{interval end},
		\emph{due date},
		and a \emph{vehicle index}.
		The vehicle index is optional, i.e. it might be~$-1$.
\end{itemize}


\end{document}



